\documentclass[12pt,a4paper,article]{memoir} 
\usepackage{geometry} % see geometry.pdf on how to lay out the page. There's lots.
\usepackage{amssymb}
\usepackage{yhmath}
\usepackage{amsmath}
\usepackage{tkz-euclide}
%Set the font (output) encoding
%--------------------------------------
\usepackage[T1]{fontenc} %Not needed by LuaLaTeX or XeLaTeX

%French-specific commands
%--------------------------------------
\usepackage[french]{babel}
\usepackage[autolanguage]{numprint} % for the \nombre command
\geometry{a4paper} % or letter or a5paper or ... etc
% \geometry{landscape} % rotated page geometry

\usepackage{etoolbox} % for \patchcmd

% make \tkzDrawLine compatible with the babel TikZ library
\makeatletter
\patchcmd{\tkz@DrawLine}{\begingroup}{\begingroup\makeatletter}{}{}
\makeatother

\usetikzlibrary{babel, angles}

% the macro is from: https://tex.stackexchange.com/questions/53947/typesetting-coordinates
\usepackage{xparse}
\ExplSyntaxOn
\NewDocumentCommand{\coord}{sO{}m}
 {
  \IfBooleanTF{#1}
   {\left(\coord_print:n {#3}\right)}
   {\mathopen{#2(}\coord_print:n {#3}\mathclose{#2)}}
 }

\seq_new:N \l_coord_list_seq
\tl_new:N \l_coord_last_tl
\cs_new_protected:Npn \coord_print:n #1
 {
  \seq_set_split:Nnn \l_coord_list_seq { , } { #1 }
  \seq_pop_right:NN \l_coord_list_seq \l_coord_last_tl
  \seq_map_inline:Nn \l_coord_list_seq { ##1 , }
  \tl_use:N \l_coord_last_tl
 }
\ExplSyntaxOff

% See the ``Article customise'' template for come common customisations
\title{Olympiades Panafricaines Mathématiques 2023 \newline Quelques solutions}
\author{KouakouSchool}
\date{\today} % delete this line to display the current date

%%% BEGIN DOCUMENT
\begin{document}

\maketitle
\tableofcontents

\vspace{3 cm}

\begin{abstract}
Ce document présente des solutions aux différents problèmes rencontrés lors des Olympiades Panafricaines de Mathématiques (OPAM) 2023.
\end{abstract}

\chapter{Enoncés}
\section{Jour 1}
\subsection{Problème 1}
Dans un triangle  $ABC$ tel que $AB < AC, D$ est un point du segment $[AC]$ tel que $BD=CD$. Une droite parallèle à $(BD)$ coupe le segment
 $[BC]$ en  $E$ et coupe la droite  $(AB)$ en  $F$.  $G$ est le point d'intersection des droites  $(AE)$ et  $(BD)$ par  $G$. 
 \begin{equation}
 \textrm{Montrer que} \quad  \widehat{BCG} = \widehat{BCF}.
 \label{question-pb-1}
 \end{equation}
 
\subsection{Problème 2}
Trouver tous les nombres entiers naturels non nuls $m$ et $n$ qui n'ont pas de diviseur commun plus grand que $1$ tels que :
\begin{equation}
m^3 + n^3 \quad \textrm{divise} \quad m^2 + 20mn + n^2.
\label{equation-pb-2}
\end{equation}

\subsection{Problème 3}
On considère la suite de nombres réels définie par: 
\begin{equation}
\left\{
	\begin{array}{l}
	x_1 = c \\
	x_{n+1} = cx_{n} + \sqrt{c^2 - 1}\sqrt{x_{n}^2 - 1}  \quad \textrm{pour tout} \quad n \geq 1.
	\end{array}
\right.
\label{equation-pb-3}
\end{equation}

Montrer que si $c$ est un nombre entier naturel non nul, alors $x_{n}$ est un entier pour tout $n \geq 1$.

\section{Jour 2}
\subsection{Problème 4}
Manzi possède $n$ timbres et un album avec $10$ pages. Il distribue les $n$ timbres dans l'album de sorte que chaque page contienne un nombre distinct de timbres. Il trouve que, peu importe comment il fait cela, il y a toujours un ensemble de $4$ pages tels que le nombre total de timbres dans ces $4$ pages soit au moins $\frac{n}{2}$.
\begin{equation}
\textrm{Déterminer la valeur maximale possible de} \quad n.
\label{question-pb-4}
\end{equation}

\subsection{Problème 5}
Soient $a$ et $b$ des nombres réels avec $a \neq 0$. Soit:
\begin{equation}
P(x) = ax^4 - 4ax^3 + (5a + b)x^2 - 4bx + b
\label{equation-pb-5}
\end{equation}
Montrer que toutes les racines de $P(x)$ sont réelles et strictement positives si et seulement si $a=b$.

\subsection{Problème 6}
Soit $ABC$ un triangle dont tous les angles sont aigus avec $AB < AC$. Soient $D, E$ et $F$ les pieds des perpendiculaires issues de $A, B$ et $C$ aux côtés opposés, respectivement. Soit $P$ le pied de la perpendiculaire issue de $F$ sur la droite $(DE)$. La droite $(FP)$ et le cercle circonscrit au triangle $BDF$ se rencontrent encore en $Q$.

\begin{equation}
\textrm{Montrer que} \quad \widehat{PBQ} = \widehat{PAD}.
\label{question-pb-6}
\end{equation}

\chapter{Solutions}
\section{Jour 1}
\subsection{Problème 1: (taux de réussite: 22/178)}
\subsubsection{Solution 1 - Utilisation d'un repère orthonormal}

Ci-dessous se trouvent la construction géométrique de la figure résultant du problème posé. Nous avons rajouté le milieu du segment $[BC]$ qu'on appellera $O$, le point $D'$ tel que la distance $OC$ égale à la distance $OD'$ et $D'$ appartient à $[OD)$ et le point $H$, le point de $(FC)$ tel que $(GH)$ est perpendiculaire à $(BC)$.
\begin{tikzpicture}[scale=3.5]
\tkzDefPoint(0,0){O}
\tkzDefPoint(-1,0){B}
\tkzDefPoint(1,0){C}
\tkzDefPoint(0,0.8){D}
\tkzDefPoint(0,1){D'}
\tkzDefPoint(-0.5,1.2){A}
\tkzDefPoint(0.35,0){E}
\tkzDrawSegment[dashed](O,D')
\tkzDrawLine(A,E)
\tkzDrawLine(B,D)
\tkzInterLL(A,E)(B,D)
	\tkzGetPoint{G}
\tkzDrawPolygon(A,B,C)
\tkzDrawLine(A,B)
\tkzDefLine[parallel=through E](B,D) \tkzGetPoint{e}
	\tkzDrawLine(E,e)
\tkzInterLL(E,e)(A,B)
	\tkzGetPoint{F}
\tkzDefLine[perpendicular=through G](B,C) \tkzGetPoint{g}
\tkzInterLL(G,g)(F,C)
	\tkzGetPoint{H}
\tkzDrawSegment[dashed](G,H)
\tkzDrawSegment(F,C)
\tkzDrawSegment(F,E)
\tkzDrawSegment(F,B)
\tkzDrawSegment(G,C)
\tkzPicAngle["$\alpha$",draw=red,
             <->,angle eccentricity=1.2,
             angle radius=1cm](G,C,B)
\tkzPicAngle["$\beta$",draw=red,
             <->,angle eccentricity=1.2,
             angle radius=1cm](B,C,F)
\tkzLabelPoints(A,B,C,D,E,F,G,H,O)
\tkzDrawPoints(A,B,C,D,E,F,G,H,O)
\end{tikzpicture}
 
 \begin{equation}
 \textrm{Montrons que} \quad \widehat{BCG} = \widehat{BCF} \quad \textrm{c'est-à-dire que} \quad \hat{\alpha} = \hat{\beta}.
 \label{question-pb-1}
 \end{equation}

Pour cela, nous allons considérer le repère orthonormal formé par les points $O$, $C$ et $D'$. Il s'agit bien d'un repère orthonormal car la droite $(OD)$ est perpendiculaire à la droite $(OC)$ puisque $(OD)$ est la médiatrice du segment $[BC]$ ($O$ est milieu de $[BC]$ et $BD = CD$) et la distance $OD'$ est égale à la distance $OC$.

\bigskip

Les coordonnées des différents points $A, B, C, D, E$ et $O$ sont les suivants. 
\begin{equation}
O = \coord[\Big]{0, 0}
\label{coord-o}
\end{equation}

\begin{equation}
C = \coord[\Big]{1, 0}
\label{coord-c}
\end{equation}

\begin{equation}
D = \coord[\Big]{0, d}, \quad \textrm{où} \quad d \quad \textrm{est une variable}.
\label{coord-d}
\end{equation}

\begin{equation}
B = \coord[\Big]{-1, 0}
\label{coord-b}
\end{equation}

\begin{equation}
A = \coord[\Big]{-\alpha, \beta}, \quad \textrm{où} \quad 0 < \alpha < 1 \quad \textrm{et} \quad \beta > 1 \quad \textrm{sont des variables}.
\label{coord-a}
\end{equation}

\begin{equation}
E = \coord[\Big]{\gamma, 0}, \quad \textrm{où} \quad -1 < \gamma < 1 \quad \textrm{où} \quad \gamma \quad \textrm{est une variable}.
\label{coord-e}
\end{equation}

La variable $\gamma$ est indépendante des autres variables d'après l'énoncé car le point $E$ est un point quelconque du segment $[BC]$.
La relation entre $\beta$, $\alpha$ et $d$ peut être déterminée d'après la position de $D$ par rapport aux points $A$, $B$ et $C$.

\bigskip

\textit{Déterminons la relation entre $\beta$, $\alpha$ et $d$:}

\smallskip

La droite $(CD)$ a pour équation du type: $ax + by + c=0$. En remplaçant les coordonnées de $C$ et $D$ dans cette équation, l'on obtient:
\begin{equation}
(DC): \quad x + \frac{1}{d}y - 1 = 0
\end{equation}

\smallskip
Le point $A$ appartient à la droite $(DC)$, donc: $ -\alpha + \frac{1}{d} \beta - 1 = 0$. Cela implique que:

\begin{equation}
\beta = d(1 + \alpha)
\label{relation-beta-alpha}
\end{equation}

\bigskip

\textit{Déterminons les valeurs des coordonnées de $F$ et $G$}

\smallskip

Concernant le point $F$, $F\coord[\Big]{e, f}$ appartient à la droite $(AB)$ et $(FE)$ est perpendiculaire à la normale $\vec{u}$ à la droite $(BD)$.
\begin{equation}
(BD): \quad x - \frac{1}{d}y + 1 = 0
\label{equation-bd}
\end{equation}

\begin{equation}
\vec{u} \coord[\Big]{1, -\frac{1}{d}}
\label{coord-vecu}
\end{equation}

\begin{equation}
(AB): \quad x - \frac{1 - \alpha}{d(1+\alpha)}y + 1 = 0
\label{equation-ab}
\end{equation}

donc:
\[\overrightarrow{FE}.\vec{u} = 0 \quad et \quad F \in (AB)\]
\[\left\{
	\begin{array}{l}
	e - \frac{1 - \alpha}{d(1 + \alpha)}f + 1 = 0 \\
	(\gamma - e).1 + \frac{1}{d}.f = 0
	\end{array}
\right.\]

ainsi:

\[\left\{
	\begin{array}{l}
	e = \frac{\alpha\gamma - \gamma - \alpha - 1}{2\alpha} \\
	f = - \frac{d(\gamma + 1)(\alpha + 1)}{2\alpha}
	\end{array}
\right.\]

\begin{equation}
F\coord[\Big]{\frac{\alpha\gamma - \gamma - \alpha - 1}{2\alpha}, - \frac{d(\gamma + 1)(\alpha + 1)}{2\alpha}}
\label{coord-f}
\end{equation}

\smallskip
Concernant le point $G$, on a:
\[ G\coord[\Big]{g, h} \in (BD) \cap (AE) \]

\begin{equation}
(AE): \quad x + \frac{\alpha + \gamma}{d(1 + \alpha)}y - \gamma = 0
\label{equation-ae}
\end{equation}

L'équation de la droite $(BD)$ est donnée par la relation (\ref{equation-bd}). Donc:

\[\left\{
	\begin{array}{l}
	g - \frac{1}{d}h + 1 = 0 \\
	g + \frac{\alpha + \gamma}{d(1 + \alpha)}h - \gamma = 0
	\end{array}
\right.\]

ainsi:

\[\left\{
	\begin{array}{l}
	g = \frac{\alpha(\gamma - 1)}{1 + 2\alpha + \gamma} \\
	h = \frac{d(1 + \gamma)(1+\alpha)}{1 + 2\alpha + \gamma}
	\end{array}
\right.\]

\begin{equation}
G\coord[\Big]{\frac{\alpha(\gamma - 1)}{1 + 2\alpha + \gamma}, \frac{d(1 + \gamma)(1+\alpha)}{1 + 2\alpha + \gamma}}
\label{coord-g}
\end{equation}

\bigskip

Pour le point $H$, on a $H \in (FC)$ tel que $(GH) \perp (BC)$. L'abscisse de $H$ est égal à $g$ c'est-à-dire l'abscisse de $G$ car $(GH) \perp (BC)$ et $(BC)$ est l'axe des abscisses, on note donc  $H\coord[\Big]{g, i}$.

\begin{equation}
(FC): \quad x + \frac{1 - e}{f}y - 1 = 0
\label{equation-fc}
\end{equation}

Donc:
\[g + \frac{1 - e}{f}i - 1 = 0\]

$1 - e \neq 0$, sinon $g = 1$. Il s'agit d'un cas limite impossible car $G \in [BC]$. De ce fait, nous pouvons écrire que:

\[i = \frac{1 - g}{1 - e}f \]

D'où, d'après (\ref{coord-f}) et par la suite (\ref{coord-g}), on a:

\[ i = -\frac{d(\gamma + 1)(\alpha + 1)}{1 + 2\alpha + \gamma} = -h \]

Donc $G\coord[\Big]{g, h}$ et $H\coord[\Big]{g, -h}$, ainsi $CG = CH$ et $(HG) \perp (CB)$. D'où:

\[\widehat{BCG} = \widehat{OCG} = \widehat{HCO} = \widehat{HCB} = \widehat{FCB}\]

Par conséquent, nous avons montré que:

\[\widehat{BCG} = \widehat{BCF} \quad \textrm{c'est-à-dire que} \quad \hat{\alpha} = \hat{\beta}\]


\subsection{Problème 2: (taux de réussite: 20/178)}
\subsection{Problème 3: (taux de réussite: 13/178)}

\section{Jour 2}
\subsection{Problème 4: (taux de réussite: 4/178)}
\subsection{Problème 5: (taux de réussite: 17/178)}
\subsection{Problème 6: (taux de réussite: 3/178)}

\end{document}