\documentclass[12pt,a4paper,article]{memoir}
\setcounter{secnumdepth}{3}
\setcounter{tocdepth}{3}
\usepackage{geometry} % see geometry.pdf on how to lay out the page. There's lots.
\usepackage{amssymb}
\usepackage{yhmath}
\usepackage{amsmath}
\usepackage{tkz-euclide}
%Set the font (output) encoding
%--------------------------------------
\usepackage[T1]{fontenc} %Not needed by LuaLaTeX or XeLaTeX

%French-specific commands
%--------------------------------------
\usepackage[french]{babel}
\usepackage[autolanguage]{numprint} % for the \nombre command
\geometry{a4paper} % or letter or a5paper or ... etc
% \geometry{landscape} % rotated page geometry

\usepackage{etoolbox} % for \patchcmd

% make \tkzDrawLine compatible with the babel TikZ library
\makeatletter
\patchcmd{\tkz@DrawLine}{\begingroup}{\begingroup\makeatletter}{}{}
\makeatother

\usetikzlibrary{babel, angles}

% the macro is from: https://tex.stackexchange.com/questions/53947/typesetting-coordinates
\usepackage{xparse}
\ExplSyntaxOn
\NewDocumentCommand{\coord}{sO{}m}
 {
  \IfBooleanTF{#1}
   {\left(\coord_print:n {#3}\right)}
   {\mathopen{#2(}\coord_print:n {#3}\mathclose{#2)}}
 }

\seq_new:N \l_coord_list_seq
\tl_new:N \l_coord_last_tl
\cs_new_protected:Npn \coord_print:n #1
 {
  \seq_set_split:Nnn \l_coord_list_seq { , } { #1 }
  \seq_pop_right:NN \l_coord_list_seq \l_coord_last_tl
  \seq_map_inline:Nn \l_coord_list_seq { ##1 , }
  \tl_use:N \l_coord_last_tl
 }
\ExplSyntaxOff

% see: https://www.physicsread.com/latex-floor-symbol/
\newcommand{\floor}[1]{\lfloor #1 \rfloor}

% see: https://tex.stackexchange.com/questions/73862/how-can-i-make-a-clickable-table-of-contents
\usepackage{hyperref}
\hypersetup{
    colorlinks,
    citecolor=red,
    filecolor=red,
    linkcolor=red,
    urlcolor=red
}

% See the ``Article customise'' template for come common customisations
\title{Olympiades Panafricaines Mathématiques 2023 \newline Quelques solutions}
\author{KouakouSchool}
\date{\today} % delete this line to display the current date

%%% BEGIN DOCUMENT
\begin{document}

\maketitle
\tableofcontents

\vspace{3 cm}

\begin{abstract}
Ce document présente des solutions aux différents problèmes rencontrés lors des Olympiades Panafricaines de Mathématiques (OPAM) 2023.
\end{abstract}

\chapter{Enoncés}
\section{Jour 1}
\subsection{Problème 1}
Dans un triangle  $ABC$ tel que $AB < AC, D$ est un point du segment $[AC]$ tel que $BD=CD$. Une droite parallèle à $(BD)$ coupe le segment
 $[BC]$ en  $E$ et coupe la droite  $(AB)$ en  $F$.  $G$ est le point d'intersection des droites  $(AE)$ et  $(BD)$ par  $G$. 
 \begin{equation}
 \textrm{Montrer que} \quad  \widehat{BCG} = \widehat{BCF}.
 \label{question-pb-1}
 \end{equation}
 
\subsection{Problème 2}
Trouver tous les nombres entiers naturels non nuls $m$ et $n$ qui n'ont pas de diviseur commun plus grand que $1$ tels que :
\begin{equation}
m^3 + n^3 \quad \textrm{divise} \quad m^2 + 20mn + n^2.
\label{equation-pb-2}
\end{equation}

\subsection{Problème 3}
On considère la suite de nombres réels définie par: 
\begin{equation}
\left\{
	\begin{array}{l}
	x_1 = c \\
	x_{n+1} = cx_{n} + \sqrt{c^2 - 1}\sqrt{x_{n}^2 - 1}  \quad \textrm{pour tout} \quad n \geq 1.
	\end{array}
\right.
\label{equation-pb-3}
\end{equation}

Montrer que si $c$ est un nombre entier naturel non nul, alors $x_{n}$ est un entier pour tout $n \geq 1$.

\section{Jour 2}
\subsection{Problème 4}
Manzi possède $n$ timbres et un album avec $10$ pages. Il distribue les $n$ timbres dans l'album de sorte que chaque page contienne un nombre distinct de timbres. Il trouve que, peu importe comment il fait cela, il y a toujours un ensemble de $4$ pages tels que le nombre total de timbres dans ces $4$ pages soit au moins $\frac{n}{2}$.
\begin{equation}
\textrm{Déterminer la valeur maximale possible de} \quad n.
\label{question-pb-4}
\end{equation}

\subsection{Problème 5}
Soient $a$ et $b$ des nombres réels avec $a \neq 0$. Soit:
\begin{equation}
P(x) = ax^4 - 4ax^3 + (5a + b)x^2 - 4bx + b
\label{equation-pb-5}
\end{equation}
Montrer que toutes les racines de $P(x)$ sont réelles et strictement positives si et seulement si $a=b$.

\subsection{Problème 6}
Soit $ABC$ un triangle dont tous les angles sont aigus avec $AB < AC$. Soient $D, E$ et $F$ les pieds des perpendiculaires issues de $A, B$ et $C$ aux côtés opposés, respectivement. Soit $P$ le pied de la perpendiculaire issue de $F$ sur la droite $(DE)$. La droite $(FP)$ et le cercle circonscrit au triangle $BDF$ se rencontrent encore en $Q$.

\begin{equation}
\textrm{Montrer que} \quad \widehat{PBQ} = \widehat{PAD}.
\label{question-pb-6}
\end{equation}

\chapter{Solutions}
\section{Jour 1}
\subsection{Problème 1: (taux de réussite: 22/178)}
\subsubsection{Solution 1 - Utilisation d'un repère orthonormal}

Ci-dessous se trouve la construction géométrique de la figure résultant du problème posé. Nous avons rajouté le milieu du segment $[BC]$ qu'on appellera $O$, le point $D'$ tel que la distance $OC$ égale à la distance $OD'$ et $D'$ appartient à $[OD)$ et le point $H$, le point de $(FC)$ tel que $(GH)$ est perpendiculaire à $(BC)$.
\begin{tikzpicture}[scale=3.5]
\tkzDefPoint(0,0){O}
\tkzDefPoint(-1,0){B}
\tkzDefPoint(1,0){C}
\tkzDefPoint(0,0.8){D}
\tkzDefPoint(0,1){D'}
\tkzDefPoint(-0.5,1.2){A}
\tkzDefPoint(0.35,0){E}
\tkzDrawSegment[dashed](O,D')
\tkzDrawLine(A,E)
\tkzDrawLine(B,D)
\tkzInterLL(A,E)(B,D)
	\tkzGetPoint{G}
\tkzDrawPolygon(A,B,C)
\tkzDrawLine(A,B)
\tkzDefLine[parallel=through E](B,D) \tkzGetPoint{e}
	\tkzDrawLine(E,e)
\tkzInterLL(E,e)(A,B)
	\tkzGetPoint{F}
\tkzDefLine[perpendicular=through G](B,C) \tkzGetPoint{g}
\tkzInterLL(G,g)(F,C)
	\tkzGetPoint{H}
\tkzDrawSegment[dashed](G,H)
\tkzDrawSegment(F,C)
\tkzDrawSegment(F,E)
\tkzDrawSegment(F,B)
\tkzDrawSegment(G,C)
\tkzPicAngle["$\alpha$",draw=red,
             <->,angle eccentricity=1.2,
             angle radius=1cm](G,C,B)
\tkzPicAngle["$\beta$",draw=red,
             <->,angle eccentricity=1.2,
             angle radius=1cm](B,C,F)
\tkzLabelPoints(A,B,C,D,E,F,G,H,O)
\tkzDrawPoints(A,B,C,D,E,F,G,H,O)
\end{tikzpicture}
 
 \begin{equation}
 \textrm{Montrons que} \quad \widehat{BCG} = \widehat{BCF} \quad \textrm{c'est-à-dire que} \quad \hat{\alpha} = \hat{\beta}.
 \label{question-pb-1-sol-1}
 \end{equation}

Pour cela, nous allons considérer le repère orthonormal formé par les points $O$, $C$ et $D'$. Il s'agit bien d'un repère orthonormal car la droite $(OD)$ est perpendiculaire à la droite $(OC)$ puisque $(OD)$ est la médiatrice du segment $[BC]$ ($O$ est milieu de $[BC]$ et $BD = CD$) et la distance $OD'$ est égale à la distance $OC$.

\bigskip

Les coordonnées des différents points $A, B, C, D, E$ et $O$ sont les suivants. 
\begin{equation}
O = \coord[\Big]{0, 0}
\label{coord-o}
\end{equation}

\begin{equation}
C = \coord[\Big]{1, 0}
\label{coord-c}
\end{equation}

\begin{equation}
D = \coord[\Big]{0, d}, \quad \textrm{où} \quad d \quad \textrm{est une variable}.
\label{coord-d}
\end{equation}

\begin{equation}
B = \coord[\Big]{-1, 0}
\label{coord-b}
\end{equation}

\begin{equation}
A = \coord[\Big]{-\alpha, \beta}, \quad \textrm{où} \quad 0 < \alpha < 1 \quad \textrm{et} \quad \beta > 1 \quad \textrm{sont des variables}.
\label{coord-a}
\end{equation}

\begin{equation}
E = \coord[\Big]{\gamma, 0}, \quad \textrm{où} \quad -1 < \gamma < 1 \quad \textrm{où} \quad \gamma \quad \textrm{est une variable}.
\label{coord-e}
\end{equation}

La variable $\gamma$ est indépendante des autres variables d'après l'énoncé car le point $E$ est un point quelconque du segment $[BC]$.
La relation entre $\beta$, $\alpha$ et $d$ peut être déterminée d'après la position de $D$ par rapport aux points $A$, $B$ et $C$.

\bigskip

\textit{Déterminons la relation entre $\beta$, $\alpha$ et $d$:}

\smallskip

La droite $(CD)$ a pour équation du type: $ax + by + c=0$. En remplaçant les coordonnées de $C$ et $D$ dans cette équation, l'on obtient:
\begin{equation}
(DC): \quad x + \frac{1}{d}y - 1 = 0
\end{equation}

\smallskip
Le point $A$ appartient à la droite $(DC)$, donc: $ -\alpha + \frac{1}{d} \beta - 1 = 0$. Cela implique que:

\begin{equation}
\beta = d(1 + \alpha)
\label{relation-beta-alpha}
\end{equation}

\bigskip

\textit{Déterminons les valeurs des coordonnées de $F$ et $G$}

\smallskip

Concernant le point $F$, $F\coord[\Big]{e, f}$ appartient à la droite $(AB)$ et $(FE)$ est perpendiculaire à la normale $\vec{u}$ à la droite $(BD)$.
\begin{equation}
(BD): \quad x - \frac{1}{d}y + 1 = 0
\label{equation-bd}
\end{equation}

\begin{equation}
\vec{u} \coord[\Big]{1, -\frac{1}{d}}
\label{coord-vecu}
\end{equation}

\begin{equation}
(AB): \quad x - \frac{1 - \alpha}{d(1+\alpha)}y + 1 = 0
\label{equation-ab}
\end{equation}

donc:
\[\overrightarrow{FE}.\vec{u} = 0 \quad et \quad F \in (AB)\]
\[\left\{
	\begin{array}{l}
	e - \frac{1 - \alpha}{d(1 + \alpha)}f + 1 = 0 \\
	(\gamma - e).1 + \frac{1}{d}.f = 0
	\end{array}
\right.\]

ainsi:

\[\left\{
	\begin{array}{l}
	e = \frac{\alpha\gamma - \gamma - \alpha - 1}{2\alpha} \\
	f = - \frac{d(\gamma + 1)(\alpha + 1)}{2\alpha}
	\end{array}
\right.\]

\begin{equation}
F\coord[\Big]{\frac{\alpha\gamma - \gamma - \alpha - 1}{2\alpha}, - \frac{d(\gamma + 1)(\alpha + 1)}{2\alpha}}
\label{coord-f}
\end{equation}

\smallskip
Concernant le point $G$, on a:
\[ G\coord[\Big]{g, h} \in (BD) \cap (AE) \]

\begin{equation}
(AE): \quad x + \frac{\alpha + \gamma}{d(1 + \alpha)}y - \gamma = 0
\label{equation-ae}
\end{equation}

L'équation de la droite $(BD)$ est donnée par la relation (\ref{equation-bd}). Donc:

\[\left\{
	\begin{array}{l}
	g - \frac{1}{d}h + 1 = 0 \\
	g + \frac{\alpha + \gamma}{d(1 + \alpha)}h - \gamma = 0
	\end{array}
\right.\]

ainsi:

\[\left\{
	\begin{array}{l}
	g = \frac{\alpha(\gamma - 1)}{1 + 2\alpha + \gamma} \\
	h = \frac{d(1 + \gamma)(1+\alpha)}{1 + 2\alpha + \gamma}
	\end{array}
\right.\]

\begin{equation}
G\coord[\Big]{\frac{\alpha(\gamma - 1)}{1 + 2\alpha + \gamma}, \frac{d(1 + \gamma)(1+\alpha)}{1 + 2\alpha + \gamma}}
\label{coord-g}
\end{equation}

\bigskip

Pour le point $H$, on a $H \in (FC)$ tel que $(GH) \perp (BC)$. L'abscisse de $H$ est égal à $g$ c'est-à-dire l'abscisse de $G$ car $(GH) \perp (BC)$ et $(BC)$ est l'axe des abscisses, on note donc  $H\coord[\Big]{g, i}$.

\begin{equation}
(FC): \quad x + \frac{1 - e}{f}y - 1 = 0
\label{equation-fc}
\end{equation}

Donc:
\[g + \frac{1 - e}{f}i - 1 = 0\]

$1 - e \neq 0$, sinon $g = 1$. Il s'agit d'un cas limite impossible car $G \in [BC]$. De ce fait, nous pouvons écrire que:

\[i = \frac{1 - g}{1 - e}f \]

D'où, d'après (\ref{coord-f}) et par la suite (\ref{coord-g}), on a:

\[ i = -\frac{d(\gamma + 1)(\alpha + 1)}{1 + 2\alpha + \gamma} = -h \]

Donc $G\coord[\Big]{g, h}$ et $H\coord[\Big]{g, -h}$, ainsi $CG = CH$ et $(HG) \perp (CB)$. D'où:

\[\widehat{BCG} = \widehat{OCG} = \widehat{HCO} = \widehat{HCB} = \widehat{FCB}\]

Par conséquent, nous avons montré que:

\[\widehat{BCG} = \widehat{BCF} \quad \textrm{c'est-à-dire que} \quad \hat{\alpha} = \hat{\beta}\]


\subsection{Problème 2: (taux de réussite: 20/178)}

\subsubsection{Solution 1 - Utilisation d'une identité remarquable}

Trouvons tous les nombres entiers naturels non nuls $m$ et $n$ qui n'ont pas de diviseur commun plus grand que $1$ tels que :
\begin{equation}
m^3 + n^3 \quad \textrm{divise} \quad m^2 + 20mn + n^2.
\label{equation-pb-2-sol-1}
\end{equation}

\bigskip

On a l'identité suivante pour tous entiers naturels $m$ et $n$:

\begin{equation}
m^3 + n^3 = (m+n)(m^2-mn+n^2)
\label{equation-factor-sum-cube}
\end{equation}

Nous allons noté le symbole $|$ comme représentant la division, c'est-à-dire $a|b$ veut dire $a$ divise $b$.
Ainsi, $m^3 + n^3$ divise $m^2+20mn+n^2$ implique que $m+n$ divise $m^2+20mn+n^2$ et $m^2-mn+n^2$ divise $m^2+20mn+n^2$. D'où l'équation:

\[\left\{
	\begin{array}{l}
	m+n | m^2+20mn+n^2 \\
	m^2-mn+n^2 | m^2+20mn+n^2
	\end{array}
\right.\]

Donc:

\[\left\{
	\begin{array}{l}
	m+n | (m+n)^2 + 18mn \\
	m^2-mn+n^2 | m^2+20mn+n^2
	\end{array}
\right.\]


Comme $m^2+20mn+n^2 - (m^2-mn+n^2) = 21mn$, en utilisant la deuxième relation, on a:
\begin{equation}
m^2-mn+n^2 | 21mn
\label{equation-divise-vgtun-prod}
\end{equation}

En utilisant la première relation et sachant que $m + n | (m + n)^2$, nous obtenons $m+n | ((m+n)^2 + 18mn) - (m+n)^2$. Cela implique:
\begin{equation}
m+n | 18mn
\label{equation-divise-sum-prod}
\end{equation}

Nous notons le symbole $\land$ comme étant la notation du plus grand diviseur commun, c'est-à-dire $a \land b$ est le plus grand diviseur commun de $a$ et $b$. Tout diviseur de $mn$ et $m+n$ est diviseur de toute combinaison linéaire. Ainsi, le plus grand diviseur commun de deux entiers naturels divise toute combinaison linéaire de ces deux entiers naturels. En l'occurence:

\[mn \land (m + n)  | mn \land mn - m*(m + n) \implies mn \land (m + n)  | mn \land - m*m\]

Or, $mn \land - m*m = m(n \land - m)$ donc:

\[mn \land (m + n)  | m(n \land - m)\]

Donc:

\begin{equation}
mn \land (m + n)  | m(n \land m)
\label{equation-pgcd-div-prod}
\end{equation}

D'après l'énoncé, $m$ et $n$ n'ont pas de diviseur commun plus grand que 1. Cela est équivalent à $m \land n = 1$.

\begin{equation}
m \land n = 1
\label{equation-pgcd-m-n}
\end{equation}

Les équations (\ref{equation-pgcd-div-prod}) et (\ref{equation-pgcd-m-n}) impliquent que:

\begin{equation}
mn \land (m + n)  | m
\label{equation-divise-sum-prod-pgcd}
\end{equation}

On a:

\[mn \land (m + n)  | m + n \]

Donc:

\[mn \land (m + n)  | (m + n) \land m \]

\[(m + n) \land m | ((m + n) - m) \land m \implies (m + n) \land m | n \land m\]

Ainsi, 

\begin{equation}
mn \land (m + n)  = 1
\label{equation-gcd-prod-sum}
\end{equation}

Les équations (\ref{equation-divise-sum-prod}) et (\ref{equation-gcd-prod-sum}) impliquent que:

\begin{equation}
m+n | 18
\label{equation-sum-div-18}
\end{equation}

D'où $m + n \in \{2, 3, 6, 9, 18\}$ car $m$ et $n$ étant des entiers naturels non nuls, $m + n \geq 2$. Afin de déterminer $m$ et $n$, nous considérons le cas où $m \geq n$ car $m$ et $n$ sont interchangeables. Ainsi, pour chaque valeur de $m+n$, nous déterminons les valeurs de $m$ et $n$ tels que $m \land n = 1$.
Cela donne:

\[m + n = 2 \implies (m, n) = (1, 1)\]
\[m + n = 3 \implies (m, n) = (2, 1)\]
\[m + n = 6 \implies (m, n) = (5, 1)\]
\[m + n = 9 \implies (m, n) \in \{(8, 1), (7, 2), (5, 4)\}\]
\[m + n = 18 \implies (m, n) \in \{(17, 1), (13, 5), (11, 7)\}\]

Donc, avec $m \geq n$, on a: 

\[(m, n) \in \{(1,1), (2, 1), (5, 1), (8, 1), (7, 2), (5, 4), (17, 1), (13, 5), (11, 7)\} \]

Pour chaque valeur de $(m, n)$ nous vérifions si l'équation (\ref{equation-divise-vgtun-prod}) est vraie. Ainsi, le tableau suivant permet de voir les résultats.

\bigskip

\begin{tabular}{|c|c|c|c|c|c|}
\hline 
$m + n$ & $m$ & $n$ & $m^2 - mn + n^2$ & $21mn$ & $m^2 - mn + n^2 | 21mn$\\ \hline                        
$2$ & $1$ & $1$ & $1$ &  $21$ & vrai \\
$3$ & $2$ & $1$ & $3$ &  $21*2$ & vrai \\
$6$ & $5$ & $1$ & $21$ &  $21*5$ & vrai \\
$9$ & $8$ & $1$ & $57=3*19$ &  $21*8*1$ & faux \\
$9$ & $7$ & $2$ & $39=3*13$ &  $21*7*2$ & faux \\
$9$ & $5$ & $4$ & $21=7*3$ &  $21*5*4$ & vrai \\
$18$ & $17$ & $1$ & $273=3*7*13$ &  $21*17$ & faux \\
$18$ & $13$ & $5$ & $129=3*43$ &  $21*13*5$ & faux \\
$18$ & $11$ & $7$ & $93=3*31$ &  $21*11*7$ & faux \\
\hline
\end{tabular}

\bigskip

Donc:

\begin{equation}
(m, n) \in \{(1,1), (2, 1), (1, 2), (5, 1), (1, 5), (5, 4), (4, 5) \}
\label{equation-pb-2-sol-1-impl}
\end{equation}

\bigskip

Réciproquement, $(m,n) \in \{(1,1), (2, 1), (1, 2), (5, 1), (1, 5)\}$ est tel que $m^3+n^3$ divise $m^2+20mn+n^2$. Cependant, 
si $(m,n) \in \{(5, 4), (4, 5)\}$ alors $m^3+n^3$ ne divise pas $m^2+20mn+n^2$. Cela se voit d'après le tableau suivant:

\bigskip

\begin{tabular}{|c|c|c|c|c|}
\hline 
$m$ & $n$ & $m^3 + n^3$ & $m^2 + 20mn + n^2$ & $m^3+n^3 | m^2 + 20mn + n^2$\\ \hline                        
$1$ & $1$ & $2$ & $22$ & vrai \\
$2$ & $1$ & $9$ & $45$ & vrai \\
$5$ & $1$ & $126$ & $126$ & vrai \\
$5$ & $4$ & $189=9*3*7$ & $441=9*7*7$ & faux \\
\hline
\end{tabular}

\bigskip

Ainsi, nous pouvons conclure que les valeurs des entiers naturels non nuls $m$ et $n$ qui n'ont pas de diviseur commun plus grand que $1$ tels que: $m^3 + n^3$ divise $m^2+20mn+n^2$ sont: 

\bigskip

\begin{equation}
(1,1), (2, 1), (1, 2), (5, 1), (1, 5)
\label{equation-pb-2-sol-1-res}
\end{equation}


\subsection{Problème 3: (taux de réussite: 13/178)}
Pour la suite de nombres réels suivante:
\begin{equation}
\left\{
	\begin{array}{l}
	x_1 = c \\
	x_{n+1} = cx_{n} + \sqrt{c^2 - 1}\sqrt{x_{n}^2 - 1}  \quad \textrm{pour tout} \quad n \geq 1.
	\end{array}
\right.
\label{equation-pb-3-sol}
\end{equation}

Montrons que si $c$ est un nombre entier naturel non nul, alors $x_{n}$ est un entier pour tout $n \geq 1$.
\subsubsection{Solution 1 - Changement de variable $x_n = cosh(a_n)$}
Définissons la suite numérique $a_n$ telle que pour tout entier naturel $n$,
\[x_n = cosh(a_n)\]
Cette suite est bien définie car la fonction $cosh$ est bijective sur $\mathbb{R^{+}}$.
En effet, la fonction cosinus hyperbolique est strictement croissante sur $\mathbb{R^{+}}$ (car sa dérivée est la fonction sinus hyperbolique qui est strictement positive sur $\mathbb{R^{+}}$), elle est donc bijective sur $\mathbb{R^{+}}$.

\bigskip

Ainsi, l'équation (\ref{equation-pb-3-sol}) devient:
\begin{equation}
\left\{
	\begin{array}{l}
	x_n = cosh(a_n) \quad \textrm{pour tout} \quad n \geq 1\\
	cosh(a_{n+1}) = cosh(a_1)cosh(a_{n}) + \sqrt{cosh(a_1)^2 - 1}\sqrt{cosh(a_{n})^2 - 1}  \quad \textrm{pour tout} \quad n \geq 1.
	\end{array}
\right.
\label{equation-pb-3-sol-cosh}
\end{equation}

Nous avons les identités remarquables suivantes pour les fonctions $cosh$ (cosinus hyperbolique) et $sinh$ (sinus hyperbolique):

\begin{equation}
\left\{
	\begin{array}{l}
	cosh(x)^2 + sinh(x)^2 = 1 \quad \textrm{pour tout} \quad x \quad \textrm{et} \quad y \quad \textrm{nombres réels}\\
	cosh(x+y) = cosh(x)cosh(y) +  sinh(x)sinh(y) \quad \textrm{pour tout} \quad x \quad \textrm{et} \quad y \quad \textrm{nombres réels}\\
	cosh(x) > 0 \quad \textrm{et} \quad sinh(x) > 0 \quad \textrm{pour tout} \quad x \quad \textrm{et} \quad y \quad \textrm{nombres réels strictement positifs}
	\end{array}
\right.
\label{equation-pb-3-relation-cosh}
\end{equation}

La seconde relation de l'équation (\ref{equation-pb-3-sol-cosh}) permet de dire:

\[cosh(a_{n+1}) = cosh(a_1)cosh(a_{n}) + sinh(a_1)sinh(a_{n}) \]

Cela implique d'après (\ref{equation-pb-3-relation-cosh}) que:

\begin{equation}
cosh(a_{n+1}) = cosh(a_1 + a_n)
\label{equation-pb-3-relation-cosh-an}
\end{equation}

La fonction cosinus hyperbolique est bijective. Ainsi l'équation (\ref{equation-pb-3-relation-cosh-an}) implique que:

\begin{equation}
a_{n+1} = a_1 + a_n
\label{equation-pb-3-relation-an}
\end{equation}

Par récurrence sur $n$, nous montrons que $a_{n} = n*a_{1}$ pour tout $n \geq 1$. En effet, pour $n=1$, cela est vrai car $a_{1} = 1*a_{1}$. Supposons que pour un certain $n \geq 1$, $\forall k \leq n$, $a_{k} = k*a_{1}$. Alors, on a:

\[a_{n+1} = a_{n} + a_{1} \implies a_{n+1} = n*a_{1} + a_{1}\]

Donc:

\[a_{n+1} = (n+1)a_{1} \]

Ainsi, par récurrence:

\begin{equation}
a_{n} = n*a_{1} \quad \textrm{pour tout} \quad n \geq 1 \quad \textrm{entier naturel}.
\label{equation-pb-3-reccur-an}
\end{equation}

Donc:

\begin{equation}
\left\{
	\begin{array}{l}
cosh(a_{1}) = c\\
x_{n} = cosh(na_{1}) \quad \textrm{pour tout} \quad n \geq 1 \quad \textrm{entier naturel}.
\end{array}
\right.
\label{equation-pb-3-reccur-xn}
\end{equation}

\bigskip

\textit{Montrons que si $cosh(x)$ est un entier alors pour tout entier naturel $n$, $cosh(nx)$ est un entier naturel.}

\bigskip

Pour tous nombres réels $a$ et $b$ et tout entier naturels $n$, on a:

\[(a + b)^n = \sum_{k=0}^{n} c_{k}^{n} a^{k}b^{n-k}\]

Car lorsqu'on choisit $a$ dans $k$ termes $a + b$, l'on choisira automatiquement $b$, $n-k$ fois pour former un terme de $(a+b)^n$. De plus, nous pouvons calculer la valeur de $c_{k}^{n}$ comme étant le nombre de manière de choisir $k$ éléments parmi $n$ éléments sans tenir compte de l'ordre, c'est à dire:

\[c_{k}^{n} = \binom{n}{k} \]

Or 

\[\binom{n}{n-k} = \binom{n}{k} \implies c_{k}^{n} = c_{n-k}^{n}\]

D'où:

\begin{equation}
(a + b)^n  = \sum_{k=0}^{\floor{\frac{n}{2}}} c_{k}^{n} (a^{k}b^{n-k} + a^{n-k}b^{k}) - \delta_{\floor{\frac{n}{2}}, \frac{n}{2}} c_{\floor{\frac{n}{2}}}^{n} a^{\frac{n}{2}}b^{\frac{n}{2}}
\label{equation-pb-3-sumabpown-2}
\end{equation}
 
 Où la notation $\floor{x}$ signifie, la partie entière de $x$ et $\delta_{a,b}$ est telle que:
 
 \[\delta_{a, b} = \left\{
	\begin{array}{l}
	1 \quad \textrm{si} \quad a = b\\
	0 \quad \textrm{si} \quad a \neq b
	\end{array}
\right.\]

Ainsi, pour $a = e^{x}$ et $b = e^{-x}$, on a: $2^{n}cosh(x)^{n} = (a + b)^{n}$. Ce qui donne donc:

\[
2^{n}cosh(x)^{n} = \sum_{k=0}^{\floor{\frac{n}{2}}} c_{k}^{n} (e^{(n-2k)x} + e^{-(n-2k)x}) - \delta_{(\floor{\frac{n}{2}}, \frac{n}{2})} c_{\frac{n}{2}}^{n}
\]

Donc:

\[
2^{n}cosh(x)^{n} = \sum_{k=0}^{\floor{\frac{n}{2}}} c_{k}^{n} 2cosh((n-2k)x) - \delta_{(\floor{\frac{n}{2}}, \frac{n}{2})} c_{\frac{n}{2}}^{n}
\]

\[
2c_{0}^{n} cosh(nx) = 2^{n}cosh(x)^{n} - \sum_{k=1}^{\floor{\frac{n}{2}}} 2c_{k}^{n}cosh((n-2k)x) + \delta_{(\floor{\frac{n}{2}}, \frac{n}{2})} c_{\frac{n}{2}}^{n}
\]

Or $c_{0}^{n} = \binom{n}{0} = 1$, donc:

\begin{equation}
cosh(nx) = 2^{n-1}cosh(x)^{n} - \sum_{k=1}^{\floor{\frac{n}{2}}} c_{k}^{n} cosh((n-2k)x) + \delta_{(\floor{\frac{n}{2}}, \frac{n}{2})} \frac{c_{\frac{n}{2}}}{2}
\label{equation-pb-3-coshnx-decomp}
\end{equation}

Soit $x$ tel que $cosh(x)$ est un entier naturel.
Par récurrence, montrons que pour tout entier naturel $n$, $cosh(nx)$ est un entier naturel. 
Pour $n = 1$, on a $cosh(nx)=cosh(x)$ est un entier naturel.

\begin{equation}
\textrm{Soit} \quad n, \quad \textrm{tel que pour tout} \quad k \leq n, \quad cosh(kx) \quad \textrm{est un entier naturel.}
\label{equation-pb-3-coshnx-hypothese}
\end{equation}

Donc:

\begin{equation}
\forall k \leq \floor{\frac{n+1}{2}}, \quad c_{k}^{n+1} cosh((n+1-2k)x) \quad \textrm{est un entier naturel}
\label{equation-pb-3-forall-coshlx}
\end{equation}

Car $\forall k \quad \textrm{tel que,} \quad 1 \leq k \leq \floor{\frac{n+1}{2}}$, $0 \leq n+1-2k \leq n$, l'hypothèse (\ref{equation-pb-3-coshnx-hypothese}) est applicable pour $l=n+1-2k$ et $c_{k}^{n+1} = \binom{n + 1}{k}$ est un entier. 

Aussi, 

\[\delta_{(\floor{\frac{n+1}{2}}, \frac{n+1}{2})} \frac{c_{\frac{n+1}{2}}^{n+1}}{2} = \left\{
	\begin{array}{l}
	\frac{c_{\frac{n+1}{2}}^{n+1}}{2} \quad \textrm{si} \quad n+1 \quad \textrm{est pair}\\
	0 \quad \textrm{si} \quad n+1 \quad \textrm{est impair}
	\end{array}
\right.\]

Si $n+1$ est pair,

\[c_{\frac{n+1}{2}}^{n+1} = \binom{n+1}{\frac{n+1}{2}} \implies c_{\frac{n+1}{2}}^{n+1} = \frac{(n+1)!}{(\frac{n+1}{2})!(\frac{n+1}{2})!}  \implies c_{\frac{n+1}{2}}^{n+1} = 2 \frac{n!}{(\frac{n-1}{2})!(\frac{n+1}{2})!} \]

\[c_{\frac{n+1}{2}}^{n+1} = 2\binom{n}{\frac{n+1}{2}} \implies c_{\frac{n+1}{2}}^{n+1} = 2c_{\frac{n+1}{2}}^{n} \implies \frac{c_{\frac{n+1}{2}}^{n+1}}{2} = c_{\frac{n+1}{2}}^{n} \]

Donc: 

\begin{equation}
\delta_{(\floor{\frac{n+1}{2}}, \frac{n+1}{2})} \frac{c_{\frac{n+1}{2}}^{n+1}}{2} = \left\{
	\begin{array}{l}
	c_{\frac{n+1}{2}}^{n} \quad \textrm{si} \quad n+1 \quad \textrm{est pair}\\
	0 \quad \textrm{si} \quad n+1 \quad \textrm{est impair}
	\end{array}
\right.
\label{equation-pb-3-cfloor-half-nplusone}
\end{equation}

D'après (\ref{equation-pb-3-coshnx-decomp}),

\[cosh((n+1)x) = 2^{n}cosh(x)^{n+1} - \sum_{k=1}^{\floor{\frac{n+1}{2}}} c_{k}^{n+1} cosh((n+1-2k)x) + \delta_{(\floor{\frac{n+1}{2}}, \frac{n+1}{2})} \frac{c_{\frac{n+1}{2}}}{2}\]

et 

\[cosh((n+1)x) > 0 \quad \textrm{pour tout} \quad x \quad \textrm{nombres réel strictement positifs}\]

En utilisant les remarques (\ref{equation-pb-3-forall-coshlx}), (\ref{equation-pb-3-cfloor-half-nplusone}) et le fait que $cosh(x)$ est un entier naturel alors $cosh((n+1)x)$ est un entier.

\bigskip

Ainsi, si $x$ est un nombre réel tel que $cosh(x)$ est un entier naturel alors pour tout entier naturel $n$, $cosh(nx)$ est un entier naturel.

\bigskip

Pour conclure, en utilisant l'expression de $x_{n}$ en fonction de $a_{1}$, (\ref{equation-pb-3-reccur-xn}) et l'implication précédente, nous pouvons déduire que:

\bigskip

Si $c$ est un nombre entier naturel non nul, alors $x_{n}$ est un entier pour tout $n \geq 1$.
 
\subsubsection{Solution 2 - Montrer $\forall n \geq 2$, $x_{n+1} = 2cx_{n} - x_{n-1}$}

\bigskip

\textit{Montrons par récurrence que pour tout entier naturel $n \geq 2$, $x_{n+1} = 2cx_{n} - x_{n-1}$}

\bigskip

Pour $n = 1, 2$ et $3$, on a:
\[x_{1} = c \]
\[x_{2} = c^2 + \sqrt{c^2 - 1}\sqrt{c^2 - 1} \implies x_{2} = 2c^2 - 1 \]
\[x_{3} = c(2c^2 - 1) + \sqrt{c^2 - 1}\sqrt{(2c^2 -  1)^2 - 1} \implies x_{3} = 2c^3 - c + \sqrt{c^2 - 1}\sqrt{4c^2(c^2 - 1)} \]
\[x_{3} = 2c^3 - c + (c^2 - 1)(2c) \implies x_{3} = 4c^3 - 3c \implies x_{3} = 2c(2c^2 - 1) - c \]

\begin{equation}
\left\{
	\begin{array}{l}
	x_{1} = c\\
	x_{2} = 2c^2 - 1\\
	x_{3} = 2c(2c^2 - 1) - c
	\end{array}
\right.
\label{equation-pb-3-sol-2-x1x2x3}
\end{equation}

D'où:

\begin{equation}
x_{3} = 2cx_{2} - x_{1}
\label{equation-pb-3-sol-2-x3fctx1x2}
\end{equation}

Donc pour $n=2$, on a: $x_{n+1} = 2cx_{n} - x_{n-1}$.

\bigskip

Supposons que pour un certain $n \geq 2$, on a: $\forall k, 2 \leq k \leq n, x_{k+1} = 2cx_{k} - x_{k-1}$. Alors, on a d'après la définition de la suite $x_{n}$,

\begin{equation}
x_{n+2} = cx_{n+1} + \sqrt{c^2 - 1}\sqrt{x_{n+1}^2 - 1}
\label{equation-pb-3-sol-2-xnplustwofctxnplusone}
\end{equation}

D'après l'hypothèse, on a:

\[x_{n+1}^2 = (2cx_{n} - x_{n-1})^2 \implies x_{n+1}^2 = 4c^{2}x_{n}^{2} - 4cx_{n}x_{n-1} + x_{n-1}^{2} \]
\[x_{n+1}^2 - 1 = (4c^{2}x_{n}^{2} - 4cx_{n}x_{n-1}) + (x_{n-1}^{2} - 1) \]
\[(c^2 - 1)(x_{n+1}^2 - 1) = (c^{2} - 1)(4c^{2}x_{n}^{2} - 4cx_{n}x_{n-1}) + (c^{2} - 1)(x_{n-1}^{2} - 1) \]

Or, d'après la définition de la suite $x_{n}$ selon l'équation (\ref{equation-pb-3-sol}):

\[(c^{2} - 1)(x_{n-1}^{2} - 1) = (x_{n} - cx_{n-1})^{2} \]

Donc:

\[(c^2 - 1)(x_{n+1}^2 - 1) = (c^{2} - 1)(4c^{2}x_{n}^{2} - 4cx_{n}x_{n-1}) + (x_{n}^{2} - 2c x_{n}x_{n-1} + c^{2}x_{n-1}^{2}) \]
\[(c^2 - 1)(x_{n+1}^2 - 1) = (4c^{4} - 4c^{2} + 1)x_{n}^{2} - 2c(2c^{2} - 1)x_{n-1}x_{n} + c^{2}x_{n-1}^{2} \]

Aussi:

\[(cx_{n+1} - x_{n})^{2} = c^{2}x_{n+1}^{2} - 2cx_{n+1}x_{n} + x_{n}^{2} \]

D'après l'hypothèse de la récurrence, pour $k=n$:

\[x_{n+1} = 2cx_{n} - x_{n-1} \quad \textrm{et} \quad x_{n+1}^2 = 4c^{2}x_{n}^{2} - 4cx_{n}x_{n-1} + x_{n-1}^{2} \]

Donc:

\[(cx_{n+1} - x_{n})^{2} = c^{2}(4c^{2}x_{n}^{2} - 4cx_{n}x_{n-1} + x_{n-1}^{2}) - 2c(2cx_{n} - x_{n-1})x_{n} + x_{n}^{2} \]
\[(cx_{n+1} - x_{n})^{2} = (4c^{4} - 4c^{2} + 1)x_{n}^{2} - 2c(2c^{2} - 1)x_{n-1}x_{n} + c^{2}x_{n-1}^{2} \]

D'où:

\[(c^2 - 1)(x_{n+1}^2 - 1) = (cx_{n+1} - x_{n})^{2}\]

L'équation (\ref{equation-pb-3-sol-2-xnplustwofctxnplusone}) donne:
\[x_{n+2} = cx_{n+1} + (cx_{n+1} - x_{n}) \quad \textrm{car} \quad cx_{n+1} \geq c^{2}x_{n} \geq x_{n} \quad (\textrm{car} \quad c \geq 1)\]
\[x_{n+2} = 2cx_{n+1} - x_{n}\]

Ainsi, par récurrence, pour tout $n \geq 2$: 

\begin{equation}
x_{n+1} = 2cx_{n} - x_{n-1}
\label{equation-pb-3-sol-2-xnplusonefct-xn-xnminusone}
\end{equation}

\bigskip

\textit{Montrons par récurrence sur $n$ que si $c$ est un entier naturel alors pour tout $n \geq 1, x_{n}$ est un entier naturel}

\bigskip

Supposons que $c$ est un entier naturel non nul.

\bigskip

Pour $n = 1$, $x_{1} = c$ est un entier naturel non nul. Pour $n=2$, l'expression de $x_{2}$ donnée par l'équation (\ref{equation-pb-3-sol-2-x1x2x3}) est: $x_{2}=2c^{2} - 1$. Donc $x_{2}$ est un entier naturel non nul.

Supposons que pour un certain $n \geq 2$, $\forall k, 1 \leq k \leq n, x_{k}$ est un entier naturel non nul. 

\bigskip

L'équation (\ref{equation-pb-3-sol-2-xnplusonefct-xn-xnminusone}) implique que $x_{n+1} = 2cx_{n} - x_{n-1}$. Or d'après l'hypothèse de récurrence $x_{n}$ et $x_{n-1}$ sont des entiers naturels non nuls. Donc, $x_{n+1}$ est un entier relatif. Aussi, $x_{n+1} \geq 1$ car les termes de l'expression de $x_{n+1} = cx_{n} + \sqrt{c^2 - 1}\sqrt{x_{n}^2 - 1}$ selon (\ref{equation-pb-3-sol}) sont tels que $cx_{n} \geq 1$ et $\sqrt{c^2 - 1}\sqrt{x_{n}^2 - 1} \geq 0$ car $c \geq 1$ et $x_{n} \geq 1$. D'où, $x_{n+1}$ est un entier naturel non nul.

\bigskip

Ainsi par récurrence, $\forall n \geq 1$, $x_{n}$ est un entier naturel non nul.

\bigskip

Pour conclure, si $c$ est un nombre entier naturel non nul, alors $x_{n}$ est un entier naturel pour tout $n \geq 1$.

\section{Jour 2}
\subsection{Problème 4: (taux de réussite: 4/178)}
\subsection{Problème 5: (taux de réussite: 17/178)}
\subsection{Problème 6: (taux de réussite: 3/178)}

\end{document}