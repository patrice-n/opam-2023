\documentclass[12pt,a4paper,article]{memoir} 
\usepackage{geometry} % see geometry.pdf on how to lay out the page. There's lots.
\usepackage{amssymb}
\usepackage{yhmath}
%Set the font (output) encoding
%--------------------------------------
\usepackage[T1]{fontenc} %Not needed by LuaLaTeX or XeLaTeX

%French-specific commands
%--------------------------------------
\usepackage[french]{babel}
\usepackage[autolanguage]{numprint} % for the \nombre command
\geometry{a4paper} % or letter or a5paper or ... etc
% \geometry{landscape} % rotated page geometry

% See the ``Article customise'' template for come common customisations

\title{Olympiades Panafricaines Mathématiques 2023 \newline Quelques solutions}
\author{KouakouSchool}
\date{\today} % delete this line to display the current date

%%% BEGIN DOCUMENT
\begin{document}

\maketitle
\tableofcontents

\vspace{3 cm}

\begin{abstract}
Ce document présente des solutions aux différents problèmes rencontrés lors des Olympiades Panafricaines de Mathématiques (OPAM) 2023.
\end{abstract}

\chapter{Enoncés}
\section{Jour 1}
\subsection{Problème 1: (taux de réussite: 22/178)}
Dans un triangle  $ABC$ tel que $AB < AC, D$ est un point du segment $[AC]$ tel que $BD=CD$. Une droite parallèle à $(BD)$ coupe le segment
 $[BC]$ en  $E$ et coupe la droite  $(AB)$ en  $F$.  $G$ est le point d'intersection des droites  $(AE)$ et  $(BD)$ par  $G$. 
 \begin{equation}
 \textrm{Montrer que} \quad  BCG = BCF.
 \end{equation}
 
\subsection{Problème 2: (taux de réussite: 20/178)}
Trouver tous les nombres entiers naturels non nuls $m$ et $n$ qui n'ont pas de diviseur commun plus grand que $1$ tels que :
\begin{equation}
m^3 + n^3 \quad \textrm{divise} \quad m^2 + 20mn + n^2.
\end{equation}

\subsection{Problème 3: (taux de réussite: 13/178)}
On considère la suite de nombres réels définie par: 
\begin{equation}
\left\{
	\begin{array}{l}
	x_1 = c \\
	x_{n+1} = cx_{n} + \sqrt{c^2 - 1}\sqrt{x_{n}^2 - 1}  \quad \textrm{pour tout} \quad n \geq 1.
	\end{array}
\right.
\end{equation}

Montrer que si $c$ est un nombre entier naturel non nul, alors $x_{n}$ est un entier pour tout $n \geq 1$.

\section{Jour 2}
\subsection{Problème 4: (taux de réussite: 4/178)}
Manzi possède $n$ timbres et un album avec $10$ pages. Il distribue les $n$ timbres dans l'album de sorte que chaque page contienne un nombre distinct de timbres. Il trouve que, peu importe comment il fait cela, il y a toujours un ensemble de $4$ pages tels que le nombre total de timbres dans ces $4$ pages soit au moins $\frac{n}{2}$.
\begin{equation}
\textrm{Déterminer la valeur maximale possible de} \quad n.
\end{equation}

\subsection{Problème 5: (taux de réussite: 17/178)}
Soient $a$ et $b$ des nombres réels avec $a \neq 0$. Soit:
\begin{equation}
P(x) = ax^4 - 4ax^3 + (5a + b)x^2 - 4bx + b
\end{equation}
Montrer que toutes les racines de $P(x)$ sont réelles et strictement positives si et seulement si $a=b$.

\subsection{Problème 6: (taux de réussite: 3/178)}
Soit $ABC$ un triangle dont tous les angles sont aigus avec $AB < AC$. Soient $D, E$ et $F$ les pieds des perpendiculaires issues de $A, B$ et $C$ aux côtés opposés, respectivement. Soit $P$ le pied de la perpendiculaire issue de $F$ sur la droite $(DE)$. La droite $(FP)$ et le cercle circonscrit au triangle $BDF$ se rencontrent encore en $Q$.

\begin{equation}
\textrm{Montrer que} \quad \widehat{PBQ} = \widehat{PAD}.
\end{equation}

\chapter{Solutions}
\section{Jour 1}
\subsection{Probleme 1}



\subsection{Probleme 2}
\subsection{Probleme 3}

\section{Jour 2}
\subsection{Probleme 4}
\subsection{Probleme 5}
\subsection{Probleme 6}

\end{document}